%%%%%%%%%%%%%%%%%%%%%%%%%%%%%%%%%%%%%%%%%%%%%%%%%%%%%%%
%                  File: gdsfactory.tex               %
%                  Date: January 2026                 %
%                                                     %
%     For preparing LaTeX manuscripts for submission  %
%     to CLEO 2026 (Optica meetings and conferences)  %
%                                                     %
%%%%%%%%%%%%%%%%%%%%%%%%%%%%%%%%%%%%%%%%%%%%%%%%%%%%%%%

\documentclass[letterpaper,10pt]{article}
%% if A4 paper needed, change letterpaper to A4

\usepackage{opticameet3} %% use version 3 for proper copyright statement

%% provide authormark
\newcommand\authormark[1]{\textsuperscript{#1}}

%% standard packages and arguments should be modified as needed
\usepackage{amsmath,amssymb}
\usepackage[colorlinks=true,bookmarks=false,citecolor=blue,urlcolor=blue]{hyperref}

% ADD THE FOLLOWING COUPLE LINES INTO YOUR PREAMBLE
\let\OLDthebibliography\thebibliography
\renewcommand\thebibliography[1]{
  \OLDthebibliography{#1}
  \setlength{\parskip}{0pt}
  \setlength{\itemsep}{0pt plus 0.3ex}
}

\begin{document}

\title{GDSFactory: An Open-Source Python Library for Chip Design and Simulation}

\author{Joaquin Matres,\authormark{1,*} Simon Bilodeau,\authormark{3} Niko Savola,\authormark{2,4} Marc de Cea,\authormark{2} Wai Kwan Yeung,\authormark{5} Erman Timurdogan,\authormark{7} Helge Gehring,\authormark{6} Floris Laporte,\authormark{1} Sebastian Goeldi,\authormark{1} and Troy Tamas\authormark{1}}

\address{\authormark{1}GDSFactory, 650 Castro St Ste 120 PMB 98035, Mountain View, CA 94041, USA\\
\authormark{2}Taara Co., Mountain View, CA, USA\\
\authormark{3}Department of Electrical and Computer Engineering, Princeton University, Princeton, NJ 08544, USA\\
\authormark{4}Department of Applied Physics, Aalto University, PO Box 13500, FIN-00076 Aalto, Finland\\
\authormark{5}Department of Physics, University of Oxford, Oxford, UK\\
\authormark{6}Google LLC, 1600 Amphitheatre Parkway, Mountain View, CA 94043, USA\\
\authormark{7}Lumentum, San Jose, CA, USA}

\email{\authormark{*}jmatres@gdsfactory.com}

\maketitle

\begin{abstract}
We present GDSFactory, an open-source Python library for integrated circuit design automation supporting photonics, analog, quantum, and MEMS applications. The platform provides unified workflows spanning layout design, device simulation, and circuit simulation via S-parameter analysis.
\end{abstract}

\section{Introduction}

Hardware iterations typically require months and millions of dollars, while software iterations cost hundreds of dollars and take hours. GDSFactory bridges this gap by providing a comprehensive Python API for chip development, including layout design, device simulation, circuit simulation, and verification~\cite{gdsfactory}.

Unlike constrained logic-driven electronic design flows~\cite{bogaerts2018}, integrated photonics requires freeform parametric geometries and tight integration between layout and electromagnetic simulation. GDSFactory addresses this with scriptable parametric cells (PCells), hierarchical assembly, routing, and seamless interfaces to industry-standard simulators---all in Python's extensive scientific ecosystem~\cite{awesome} (Fig.~\ref{fig:flow}).

\begin{figure}[htbp]
\centering
\includegraphics[width=8cm]{figures/flow/flow.pdf}
\caption{GDSFactory workflow: Design phase (simulation, schematic capture, layout), Verification (DRC, LVS), and Validation to ensure the fabricated product meets specifications.}
\label{fig:flow}
\end{figure}

\section{Layout Design}

GDSFactory implements cells as Python functions returning a \texttt{Component} class with polygons, port metadata, and convenience methods. Using KLayout's C++ geometry engine~\cite{klayout}, users define parametric cells through a functional programming approach where the \texttt{@gf.cell} decorator handles caching to eliminate redundant regeneration. Port metadata enables automatic routing via various routing functions that define routes between ports or groups of ports.

\section{Device Simulation}

GDSFactory's gplugins repository~\cite{gplugins} provides unified interfaces to a growing list of simulators by reusing the core layout abstractions (Components, Layerstacks, Ports). For instance, Components can be meshed via GMSH \cite{gmsh} (through a wrapper \cite{meshwell}) for cross-sectional or 3D analysis (Fig.~\ref{fig:mesh}). Finite-difference time domain electromagnetic simulation is supported through open-source backends including MEEP~\cite{meep} and Luminescent AI, as well as proprietary solvers Tidy3D~\cite{tidy3d} (cloud-based GPU acceleration) and Lumerical FDTD. FEM and multiphysics solvers include Femwell~\cite{femwell} for waveguide mode analysis and thermal simulation, Palace~\cite{palace} for RF and microwave simulations, and MEOW \cite{meow} for eigenmode expansion. TCAD simulation uses DEVSIM~\cite{devsim} for semiconductor device physics and Sentaurus for advanced process simulation.

\begin{figure}[htbp]
\centering
\includegraphics[width=8cm]{figures/gmsh/dessin.pdf}
\caption{GDSFactory meshing: (a) heater layout, (b) cross-sectional mesh, (c) 3D mesh.}
\label{fig:mesh}
\end{figure}

\section{Circuit Simulation}

Circuit-level simulation enables system-scale photonic design. GDSFactory facilitates this through netlist extraction, which enables compositions of device-level simulations. For instance, we have used SAX~\cite{sax} for differentiable S-parameter circuit simulation using JAX, supporting automatic differentiation for gradient-based optimization, Monte Carlo analysis for yield estimation, and wavelength-dependent S-parameter interpolation from FDTD results. VLSIR provides SPICE netlist export for mixed photonic-electronic simulation.

\section{Process Design Kits}

Open-source PDKs include GlobalFoundries 180nm, SkyWater 130nm~\cite{skywater}, VTT 3\,\textmu m SOI, SiEPIC, Cornerstone, IHP, Luxtelligence, and Quantum-RF-PDK. Commercial PDKs available through GDSFactory+ subscription~\cite{gdsfactoryplus} include AIM Photonics, AMF, CompoundTek, Fraunhofer HHI, Smart Photonics, Tower PH18, OpenLight, III-V Labs, LioniX, Ligentec, Lightium, and QCI. The generic PDK follows standard layer conventions~\cite{chrostowski2015} for cross-foundry compatibility.

\section{Conclusion}

GDSFactory provides a unified Python-driven workflow spanning layout, device simulation, and circuit simulation. The tight integration between device solvers and circuit simulators enables rapid design iteration from component to system level, and its programmatic, open nature enables modern agentic workflows \cite{Sharma2025}.  The library is freely available at \url{https://github.com/GDSFactory/GDSFactory}.

\begin{thebibliography}{99}

\bibitem{gdsfactory} J. Matres \textit{et al.}, ``GDSFactory,'' GitHub (2024), \url{https://github.com/GDSFactory/GDSFactory}.

\bibitem{bogaerts2018} W. Bogaerts \textit{et al.}, ``Silicon photonics circuit design: methods, tools and challenges,'' Laser Photon. Rev. \textbf{12}, 1700237 (2018).

\bibitem{awesome} J. Matres, ``Awesome Photonics,'' GitHub (2024), \url{https://github.com/joamatab/awesome_photonics}.

\bibitem{klayout} M. K\"offerlein, ``KLayout,'' \url{https://www.klayout.de/}.

\bibitem{gplugins} J. Matres \textit{et al.}, ``gplugins,'' GitHub (2024), \url{https://github.com/gdsfactory/gplugins}.

\bibitem{gmsh} C.~Geuzaine and J.-F. Remacle, ``Gmsh: {A} 3-{D} finite element mesh generator with built-in pre- and post-processing facilities,'' \emph{Int. J. Numer. Meth. Engng}, vol.~79, no.~11, pp. 1309--1331, 2009. [Online]. Available: \url{https://doi.org/10.1002/nme.2579}.

\bibitem{meshwell} S. Bilodeau \textit{et al.}, ``Meshwell,'' GitHub (2026), \url{https://github.com/simbilod/meshwell}.

\bibitem{meep} A. F. Oskooi \textit{et al.}, ``MEEP: A flexible free-software package for electromagnetic simulations by the FDTD method,'' Comput. Phys. Commun. \textbf{181}, 687--702 (2010).

\bibitem{tidy3d} Flexcompute Inc., ``Tidy3D,'' \url{https://www.flexcompute.com/tidy3d/}.

\bibitem{femwell} H. Gehring \textit{et al.}, ``Femwell,'' GitHub (2023), \url{https://github.com/HelgeGehring/femwell}.

\bibitem{palace} AWS, ``Palace: 3D Finite Element Solver for Computational Electromagnetics,'' GitHub (2024), \url{https://github.com/awslabs/palace}.

\bibitem{meow} F. Laporte, ``MEOW,'' GitHub (2024), \url{https://github.com/flaport/meow}.

\bibitem{devsim} J. E. Sanchez, ``DEVSIM,'' \url{https://devsim.org/}.

\bibitem{sax} F. Laporte, ``SAX,'' GitHub (2023), \url{https://github.com/flaport/sax}.

\bibitem{skywater} SkyWater Technology Foundry and Google, ``SkyWater Open Source PDK,'' GitHub (2023), \url{https://github.com/google/skywater-pdk}.

\bibitem{gdsfactoryplus} GDSFactory, ``GDSFactory+,'' \url{https://gdsfactory.com/}.

\bibitem{chrostowski2015} L. Chrostowski and M. Hochberg, \textit{Silicon Photonics Design} (Cambridge, 2015).

\bibitem{Sharma2025}
A.~Sharma \emph{et al.}, ``AI agents for photonic integrated circuit design automation,'' \emph{APL Machine Learning}, vol.~3, no.~4, Dec. 2025. doi: 10.1063/5.0300741.

\end{thebibliography}

\end{document}
